%----------------- KONFIGURATION ----------------- %
\pagestyle{empty} % enthalten keinerlei Kopf oder Fuß


\chapter*{Abstract} % (fold)
\label{cha:abtract}
	In der folgenden wissenschaftlichen Arbeit setze ich mich mit den Graphen Datenbanken auseinander.
	Diese unterscheiden sich grundlegend in der Speicherung und Abfrage von Daten, im Vergleich zu Relationalen Datenbanken.
	Mit ihnen können große Mengen von Beziehungen abfrage effizient und logisch abgespeichert werden.
	Graphendatenbanken erfreuen sich in den letzten jahren steigender belibtheit und Verbreitung, vorallem durch ihren Einsatz in Sozialen Netzwerken.
	Aufgrund ihrer Verbreitung und der Fähigkeit wichtige Informationen zu generieren werden sie auch in kleineren Projekten immer mehr eingesetz.
	Bei den Graphendatenbanken gibt es noch keinen festen Standart für die Abfrage und Analyse der gespeicherten Daten.
	Dieses Teilgebiet halte ich für besonders wichtig in der Auswahl des Datenbank Systems, da aus diesem ein Erhebliches Wissen gerneriert werden kann. 
	Deswegen evaluiere ich die möglichkeiten der Standatisierung und vergleiche die einzelnen Systeme auf ihre vorhandene Funktionalität.
	Ich beschränke mich bei meinem Test auf echte Graphendatenbanken, also solche die sowhohl das Speichern der Daten als auch deren Vernetzung nativ unterstützen und nicht nur simulieren.
	Ich habe das System Neo4j als vorreiter der der Graphendatenbanken und Cayley ein opensource Project von Google zum Vergleich herran gezogen.
	
	
\nocite{Robinson2015}