\documentclass[a4paper,12pt,headsepline]{report}

%----------------- PDF CONFIG ----------------- %
\pdfinfo{    
     /Title (PDF-Titel) 
     /Subject   (PDF-Thema)    
     /Author  (Vorname Nachname) 
     /Keywords   (Stichwort1,Stichwort2)      
} 

\title{MyTitle}
\author{theAuthor}
\date{1.1.2000}



%----------------- PAKETE INKLUDIEREN ----------------- %

\usepackage{geometry} % Packet für Seitenrandabständex und Einstellung für Seitenränder
\usepackage[ngerman]{babel} % deutsche Silbentrennung

\usepackage{booktabs} %entzerrt die Tabellenzeilen und bietet verschieden dicke Unterteilungslinien
\usepackage{longtable} % Tabellen können sich nicht über mehrere Seiten 
\usepackage{graphicx} % kann LaTeX Grafiken einbinden

\usepackage[applemac]{inputenc} % Umlaute unter Mac werden automatisch gesetzt
\usepackage[T1]{fontenc} % Zeichenencoding
\usepackage{lmodern} % typographische Qualität 
\frenchspacing % Schaltet den zusätzlichen Zwischenraum ab
\usepackage{fix-cm}
\usepackage{hyperref} % verwandelt alle Kapitelüberschriften, Verweise aufs Literaturverzeichnis und andere Querverweise in PDF-Hyperlinks
\usepackage{color}
\usepackage{url}


\usepackage[nottoc]{tocbibind}



% für Listings
\usepackage{listings}
\lstset{numbers=left, numberstyle=\tiny, numbersep=5pt, stepnumber=4, keywordstyle=\color{black}\bfseries\itshape, stringstyle=\ttfamily,showstringspaces=false,basicstyle=\footnotesize,captionpos=b}
\lstset{language=java}



%----------------- FARBEN DEFINIEREN ----------------- %
\definecolor{gray}{gray}{0.95} % Listingsbackground

%----------------- LAYOUT SETZEN ----------------- %
\geometry{left=2cm, right=2cm, top=2.5cm, bottom=2cm}
\linespread {1.25}\selectfont %1.25 da er von Haus aus 1.2 ist und 1,25 * 1,2 = 1,5 isch




%----------------- ANFANG INHALT ----------%
\begin{document}



\pagenumbering{roman} % Seitennummer

%----------------- DECKBLATT -----------------%
 %----------------- KONFIGURATION ----------------- %
\pagestyle{empty} % enthalten keinerlei Kopf oder Fuß 
%----------------- HDA FBI Logo ----------------- %
\begin{figure}[t]
	\centering
	\includegraphics[width=0.6\textwidth]{logo_fbi.eps}
\end{figure}

%----------------- INHALT ----------------- %

\begin{center}
\Large Hochschule Darmstadt \\
\normalsize \textsc{- Fachbereich Informatik -} \\

% Whitespace
\vspace{105 pt}

\Huge Graphen Datenbanken \\ 
\normalsize
\vspace{20 pt}

Wissenschaftliches Arbeiten 2\\ 


\vspace{75 pt}


vorgelegt von \\
\vspace{5 pt}
Jan Niklas Hollenbeck \\
735992
\vspace{115 pt}

\begin{tabular}[h]{p{4cm}l l}
	Referent: & Prof. Dr. Martin Abel\\
	Korreferent: & Prof. Dr. Andreas M\"uller
\end{tabular}


\end{center}

 
%----------------- ABSTRACT -----------------%
 %----------------- KONFIGURATION ----------------- %
\pagestyle{empty} % enthalten keinerlei Kopf oder Fuß


\chapter*{Abstract} % (fold)
\label{cha:abtract}
	In der folgenden wissenschaftlichen Arbeit setze ich mich mit den Graphen Datenbanken und ihrer Funktionsweise auseinander.
	Im Vergleich zu relationalen Datenbanken unterscheiden sich diese grundlegend in der Speicherung und Abfrage von Daten. 
	Mit ihnen können große Mengen von Vernetzten Daten abfrage effizient und logisch abgespeichert werden.
	Graphendatenbanken erfreuen sich in den letzten Jahren steigender Belibtheit und Verbreitung, begünstigt vorallem durch ihren Einsatz in Sozialen Netzwerken.
	Aufgrund ihrer wachsenden Verbreitung und der Fähigkeit wichtige Informationen zu generieren werden sie auch zunehmend in kleineren Projekten eingesetzt.
	Bei Graphendatenbanken gibt es keinen festen Standart für die Abfrage und Analyse der gespeicherten Daten.
	Dieses Teilgebiet halte ich für besonders wichtig in der Auswahl des Datenbank Systems, weil mit diesem ein Erhebliches Wissen generiert werden kann. 
Deswegen evaluiere ich die Möglichkeiten der Standartisierung von Abfragen.
Außerdem vergleiche die einzelnen Systeme auf ihre eigenen Funktionalität hinsichtlich der Analyse von Daten.
	Ich beschränke mich bei meinem praktischen Test auf echte Graphendatenbanken, also solche die sowohl das Speichern der Daten als auch deren Vernetzung nativ unterstützen und nicht nur simulieren.
	Von diesen habe ich Neo4j als Vorreiter der der Graphendatenbanken und Cayley ein aktuelles opensource Project von Google zum Vergleich heran gezogen.
 
 
%----------------- VERZEICHNISSE -----------------%
\tableofcontents % Inhaltverzeichnis

\pagestyle{plain} % zurueck setzen von roemische seitenanzahl

%----------------- EINLEITUNG -----------------%
\chapter{Einleitung}
längere Version des Abstracts 


\chapter{Graphendatenbanken}
\section{Graphen}
kurze erklärung zu Graphen und deren Funktionsweise.
\section{Funktionsweise Graphendatenbanken}
Einführung in das Konzept der Graphendatenbanken
und abgrenzung zu anderen Datenbanken
\section{Modelle}
Vorstellung der Verschiedenen Graphen Modelle zur Strukturierung der Datenbank
\section{Abfrage und Analyse von Daten}
Wie werden Daten in Graphendatenbanken abgefragt und zur Wertgewinnung benutzt.
Einfaches abfragen von Komplexen Informationen
\subsubsection{Link Analysis}
Nutzen von Graphendatenbanken zur generierung von wertvollen Infromationne
\subsection{Traversierung}


%----------------- INHALT -----------------%
\chapter{Vergleich der Graphendatenbanken}
Problem der nicht Standartisierung erläutern und die Methodik der Evaluierung und des praktischen tests veranschaulichen
\section{Standartlösungen für Querys}
Die Vor und Nachteile von Standartlösungen bei Graphendatenbanken
\section{Funktionen der Systeme}
Wer kann was und nach welchen Kriterien wird bewertet 


%----------------- ERGEBNISSE -----------------%
\chapter{Ergebnisse}
Darstellung der Ergebnisse zu den Systemen und einzelnen Themen

\chapter{Related Work}
\cite{Robinson2015}
\cite{Angles2012}
\cite{Liu2009}
\bibliographystyle{amsalpha}
\bibliography{literatur}
\pagenumbering{Roman}	
	
\end{document}
	