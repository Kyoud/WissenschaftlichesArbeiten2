%----------------- KONFIGURATION ----------------- %
\pagestyle{empty} % enthalten keinerlei Kopf oder Fuß


\chapter*{Abstract} % (fold)
\label{cha:abtract}
	In der folgenden wissenschaftlichen Arbeit setze ich mich mit den Graphen Datenbanken und ihrer Funktionsweise auseinander.
	Im Vergleich zu relationalen Datenbanken unterscheiden sich diese grundlegend in der Speicherung und Abfrage von Daten. 
	Mit ihnen können große Mengen von Vernetzten Daten abfrage effizient und logisch abgespeichert werden.
	Graphendatenbanken erfreuen sich in den letzten Jahren steigender Belibtheit und Verbreitung, begünstigt vorallem durch ihren Einsatz in Sozialen Netzwerken.
	Aufgrund ihrer wachsenden Verbreitung und der Fähigkeit wichtige Informationen zu generieren werden sie auch zunehmend in kleineren Projekten eingesetzt.
	Bei Graphendatenbanken gibt es keinen festen Standart für die Abfrage und Analyse der gespeicherten Daten.
	Dieses Teilgebiet halte ich für besonders wichtig in der Auswahl des Datenbank Systems, weil mit diesem ein Erhebliches Wissen generiert werden kann. 
Deswegen evaluiere ich die Möglichkeiten der Standartisierung von Abfragen.
Außerdem vergleiche die einzelnen Systeme auf ihre eigenen Funktionalität hinsichtlich der Analyse von Daten.
	Ich beschränke mich bei meinem praktischen Test auf echte Graphendatenbanken, also solche die sowohl das Speichern der Daten als auch deren Vernetzung nativ unterstützen und nicht nur simulieren.
	Von diesen habe ich Neo4j als Vorreiter der der Graphendatenbanken und Cayley ein aktuelles opensource Project von Google zum Vergleich heran gezogen.